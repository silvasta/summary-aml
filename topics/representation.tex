\section{Key ML Concepts }

\subsection{What is Machine Learning?}
ML enables systems to learn from data without explicit programming. Goals:
Prediction, inference, decision-making.

\textbf{Types of ML:}
\begin{itemize}
  \item \textbf{Supervised}: Labeled data; predict $y$ from $x$ (e.g.,
    regression, classification).
  \item \textbf{Unsupervised}: Unlabeled data; find patterns (e.g.,
    clustering, dimensionality reduction).
  \item \textbf{Reinforcement}: Agent learns via rewards (e.g., MDPs).
  \item \textbf{Semi-supervised/Self-supervised}: Mix of labeled/unlabeled.
\end{itemize}

\textbf{ML Pipeline:} Data collection $\to$ Preprocessing $\to$ Feature
engineering $\to$ Model selection $\to$ Training $\to$ Evaluation $\to$
Deployment.

\subsection{Empirical Risk Minimization (ERM)}
Core principle: Minimize average loss on training data as proxy for true risk.

\textbf{Formulas:}
\begin{itemize}
  \item True Risk: $R(f) = \mathbb{E}_{(x,y) \sim \mathcal{D}} [\ell(f(x),
    y)]$, where $\ell$ is loss function, $\mathcal{D}$ is data distribution.
  \item Empirical Risk: $\hat{R}(f) = \frac{1}{n} \sum_{i=1}^n \ell(f(x_i),
    y_i)$.
  \item Goal: $\hat{f} = \argmin_f \hat{R}(f)$ (often with regularization).
\end{itemize}

\textbf{Trick:} Overfitting occurs when $\hat{R} \ll R$; use validation sets,
cross-validation.

\subsection{Bias-Variance Tradeoff}
Decomposes generalization error: $\mathbb{E}[(y - \hat{y})^2] = \text{Bias}^2
+ \text{Variance} + \text{Noise}$.

\textbf{Formulas:}
\begin{itemize}
  \item Bias: $\mathbb{E}[\hat{y}] - y$ (systematic error).
  \item Variance: $\mathbb{E}[(\hat{y} - \mathbb{E}[\hat{y}])^2]$ (sensitivity
    to data).
\end{itemize}
\emph{Exam Hint:} High bias $\to$ underfitting (simple models); high variance
$\to$ overfitting (complex models). Regularization reduces variance.

\subsection{Basic Loss Functions}
Common in early exams for supervised tasks.

\begin{itemize}
  \item Regression (MSE): $\ell(y, \hat{y}) = (y - \hat{y})^2$.
  \item Classification (0-1 Loss): $\ell(y, \hat{y}) = \mathbb{I}(y \neq
    \hat{y})$.
  \item Logistic Loss: $\ell(y, p) = -y \log p - (1-y) \log(1-p)$ (for
    binary).
\end{itemize}

\textbf{Trick:} For optimization, use surrogates (e.g., hinge loss for SVM
instead of 0-1).

\subsection{Overfitting \& Regularization}
Prevent by adding penalty: $\hat{f} = \argmin_f \hat{R}(f) + \lambda
\Omega(f)$, where $\Omega$ is complexity (e.g., L2: $\|w\|^2$).

\emph{Hint:} Cross-validation for $\lambda$; exams may ask to derive
regularized ERM.

\section{Representations}

\subsection{Key Concepts \& Tricks}
\begin{itemize}
  \item \textbf{Representations}: Feature maps $\phi: \mathcal{X} \to
    \mathbb{R}^d$ for non-linear models (e.g., kernels). Exam trick: Use for
    bounds in estimation (e.g., info in embedded spaces).
  \item \textbf{CRLB}: Lower bound on variance of unbiased estimators
    $\hat{\theta}$ of param $\theta$. Assumes regularity: differentiable
    log-likelihood, finite variance.
  \item Trick: CRLB achieved iff estimator is efficient (e.g., MLE in
    exponential families). Multivariate: Use inverse Fisher matrix.
  \item Hint for exams: Always check unbiasedness first; compute via Hessian or
    score function.
\end{itemize}

\subsection{Formulas: Fisher Information}
Fisher info measures param sensitivity in likelihood. For scalar $\theta$:
\[
  I(\theta) = \mathbb{E}\left[ \left( \frac{\partial}{\partial \theta} \log
  p(X|\theta) \right)^2 \right] = -\mathbb{E}\left[ \frac{\partial^2}{\partial
  \theta^2} \log p(X|\theta) \right]
\]
Multivariate ($\theta \in \mathbb{R}^k$): Matrix form
\[
  [I(\theta)]_{ij} = \mathbb{E}\left[ \frac{\partial \log p}{\partial \theta_i}
  \frac{\partial \log p}{\partial \theta_j} \right] = -\mathbb{E}\left[
  \frac{\partial^2 \log p}{\partial \theta_i \partial \theta_j} \right]
\]
Trick: For iid samples $X_1,\dots,X_n$, $I_n(\theta) = n I(\theta)$.

\begin{sstTitleBox}[RoyalBlue]{Example: Gaussian $\mathcal{N}(\mu,
  \sigma^2=1)$}
  Score: $\frac{\partial \log p}{\partial \mu} = x - \mu$. Fisher: $I(\mu) = 1$.
\end{sstTitleBox}

\subsection{Rao-Cramér Lower Bound (CRLB)}
For unbiased estimator $\hat{\theta}(X)$:
\[
  \Var(\hat{\theta}) \geq \frac{1}{I(\theta)} \quad (\text{scalar})
\]
Multivariate (for function $g(\theta)$):
\[
  \Var(g(\hat{\theta})) \geq \left( \frac{\partial g}{\partial \theta}
  \right)^T I(\theta)^{-1} \left( \frac{\partial g}{\partial \theta} \right)
\]
General CRLB:
\[
  \Cov(\hat{\theta}) \succeq I(\theta)^{-1}
\]
Trick: Equality if $\hat{\theta} = a(\theta) \cdot s(X) + b(\theta)$, where
$s(X)$ is sufficient statistic (e.g., in Gaussians, sample mean achieves
CRLB).

\subsection{Calculus Recipes \& Derivations}
\begin{itemize}
  \item Compute $I(\theta)$: (1) Write $\log L(\theta|X) = \sum \log
    p(x_i|\theta)$. (2) Take 2nd deriv or score sq. (3) Expectation over
    $p(X|\theta)$.
  \item For representations: Info in feature space: $I_\phi(\theta) =
    \mathbb{E}[\phi(X)^T \phi(X)]^{-1}$ (e.g., for linear models).
  \item Exam hint: In ML, CRLB bounds learning rates (e.g., variance in param
    est. for neural nets).
\end{itemize}

\begin{sstOnlyFrame}
  \textbf{Regularity Conditions}: Support indep. of $\theta$; $\int p(x|\theta)
  dx =1$ differentiable under integral.
\end{sstOnlyFrame}

